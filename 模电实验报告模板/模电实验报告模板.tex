\documentclass[12pt,a4paper]{article}
\usepackage{geometry}
\geometry{left=2.5cm,right=2.5cm,top=2.5cm,bottom=2.5cm}
\usepackage{graphicx}
\usepackage{setspace}
\usepackage{array}
\usepackage{booktabs}
\usepackage{multirow}
\usepackage{amssymb}
\usepackage{indentfirst}
\usepackage{zhnumber}
\usepackage[UTF8]{ctex}

% 设置字体
\setmainfont{Times New Roman}
\setCJKmainfont{SimSun}

\setlength{\parindent}{2em} % 首行缩进2字符
\linespread{1.5} % 行间距约22磅

% 定义带有下划线的文本框命令
\newcommand{\underlinebox}[2][3cm]{\underline{\makebox[#1]{#2}}}

\title{\textbf{电路与模拟电子技术实验 实验报告}}
% 直接在模板中填写信息
\author{班级 \underlinebox[2cm]{2023} \quad 
	姓名 \underlinebox[2cm]{张三} \quad 
	学号 \underlinebox[2cm]{20230001} \quad 
	成绩 \underlinebox[1.5cm]{ }}
\date{}

\begin{document}
	
	\maketitle
	
	\begin{table}[h]
		\centering
		\renewcommand{\arraystretch}{1.5}
		\begin{tabular}{|c|c|c|c|}
			\hline
			\textbf{实验日期} & \underlinebox[3cm]{2024年3月10日} & \textbf{实验分组} & $\square$上午 $\square$下午 $\square$晚上 \\
			\hline
			\textbf{桌号} & \underlinebox[2cm]{5号} & \textbf{同组同学姓名或编号} & \underlinebox[3cm]{李四} \\
			\hline
		\end{tabular}
	\end{table}
	
\section*{\centerline{实验一 \quad Multisim与电工实验常用电子仪器的使用}}

\vspace{1em}

\section*{一、实验目的}

	% 内容自行填写


\section*{二、实验仪器和设备}

	% 内容自行填写


\section*{三、实验内容与要求}

\subsection*{(一)、直流电源+万用表 实验}

\subsubsection*{1、实验电路(仿真文件“实验1-1”)}

	% 插入电路图
	\begin{center}
		\includegraphics[width=0.8\textwidth]{示例图片.png} % 请替换为实际图片路径
	\end{center}


\subsubsection*{2、仿真实验结果}

	% 内容自行填写


\subsubsection*{3、真实实验结果}

	% 内容自行填写


\vspace{1em} % 实验之间空一行

\subsection*{(二)、信号发生器+示波器 仿真实验}

\subsubsection*{1、仿真实验电路(文件“实验1-2”)}

	% 插入电路图
	\begin{center}
		%\includegraphics[width=0.8\textwidth]{circuit2.png} % 请替换为实际图片路径
	\end{center}

\subsubsection*{2、仿真实验结果}

	% 内容自行填写


\subsubsection*{3、真实实验结果}

	% 内容自行填写


\section*{四、实验总结、收获体会和建议(包括实验出现的问题及处理方法)}

	% 内容自行填写


\section*{五、思考题}

	% 内容自行填写


\end{document}